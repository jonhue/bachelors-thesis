\chapter{\abstractname}

The energy consumption of data centers assumes a significant fraction of the worlds overall energy consumption. Most data centers are statically provisioned leading to a very low average utilization of servers. In this work we survey uni-dimensional and high-dimensional approaches for dynamically powering-up and powering-down servers to reduce the energy footprint of data centers while ensuring that incoming jobs are processed in-time. We implement algorithms for smoothed online convex optimization and variations thereof where in each round the agent receives a convex cost function. The agent seeks to minimize cost based on an exploration-exploitation trade-off using an additional switching cost which is associated with changing decisions in-between rounds. We implement the algorithms in their most general form, inviting future research on their performance in other application areas. We evaluate the algorithms for the application of right-sizing data centers using traces from Facebook, Microsoft, Alibaba, and Los Alamos National Lab. Our experiments show that the the online algorithms perform close to the dynamic offline optimum in practice and promise a significant cost reduction when compared to the static offline optimum. We discuss how features of the data center model and trace impact the performance. Finally, we investigate the practical use of predictions to achieve further cost reductions.
