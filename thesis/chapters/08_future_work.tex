% !TeX root = ../main.tex
% Add the above to each chapter to make compiling the PDF easier in some editors.

\chapter{Future Work}\label{chapter:future_work}

Our implementation separates the model layer from the problem and algorithm layer. Therefore, testing the empirical performance of the discussed online algorithms in other application areas is a natural and exciting direction for future research. In \cref{section:introduction:related_work}, we have given an overview of some promising applications.

\paragraph{Algorithms for Convex Body Chasing} We have mentioned in \autoref{chapter:introduction} that smoothed convex optimization and convex body chasing are equivalent. Therefore, an interesting research project would be to extend our library of implemented algorithms by the known algorithms for convex body chasing to compare their empirical performance. In particular, the $\mathcal{O}(d)$-competitive algorithm obtained by \citeauthor*{Argue2019}~\cite{Argue2019} is highly relevant for the application of right-sizing data centers as it does not impose a restriction on cost functions beyond their convexity.

\paragraph{Dynamic Bounds and Dimensions} In practice, the number of available servers (and even server types) in a data center is likely to change over time. An unexplored area of research is how the discussed approaches for online algorithms can be extended to a setting where the bounds on the decision space $\mathcal{X}$ and the dimension of $\mathcal{X}$ are allowed to change over time. \citeauthor*{Albers2021_2}~\cite{Albers2021_2} discuss how their offline algorithm solving the multi-dimensional integral case can be extended to a setting with time-dependent bounds.

\paragraph{Optimal Valley Filling} In \cref{section:case_studies:method:alternatives}, we have discussed the advantages and disadvantages of valley filling, i.e., scheduling low-priority tasks during periods of low loads to reduce the peak-to-mean ratio. Therefore, an interesting problem is finding optimal server configurations and job scheduling times such that the operating costs and switching costs of the data center are balanced with the revenue loss of delaying specific jobs. This problem extends smoothed convex optimization in the data center setting by allowing for incoming loads to be postponed to a later time slot.

\paragraph{Optimal Assignments of Jobs to Servers} Smoothed convex optimization determines the optimal assignment of jobs to a collection of servers of the same type. In \cref{section:application:dispatching}, we have seen that an optimal dispatching rule is to distribute jobs across all servers evenly. However, this approach does not have to be optimal in practice as job arrival times may vary and jobs are discrete, i.e., it may not be possible to distribute them evenly. Therefore, an interesting problem is determining an optimal scheduling of jobs that accounts for the approximations of smoothed convex optimization.

\paragraph{Long-Running Jobs} An vital research problem is to extend the data center model we discussed in \cref{chapter:application} to support jobs with a longer runtime than the length of a time slot. In our case studies from \cref{chapter:case_studies}, we have seen that for a time slot length on the order of tens of minutes to an hour, a significant fraction of jobs (in practical scenarios more than 50\%) fulfill this criterion. This can be achieved by ``memorizing'' which jobs have not been completed yet and ensuring that enough servers of each type are active in the following time slot so that jobs do not have to be rerouted to a server of a different type.

\paragraph{Gang Scheduling Requirement} A similar problem is to extend the model to allow for mutual job requirements. For example, requirements ensuring jobs are processed simultaneously or on servers of the same type.

\paragraph{Without Lookahead} The classical smoothed online convex optimization problem assumes that the convex cost function is known before a move has to be made. This is reasonable to separate the problem of anticipating movement costs with current hitting costs from estimating the current hitting costs. However, in practice, many applications require action before the hitting costs are known. For example, in the setting of right-sizing data centers, enough servers need to be available before jobs arrive as the powering up of servers requires some time. We have seen in \cref{section:online_algorithms:md:predictions} that predictions can be used to get around this restriction. Nonetheless, it would be interesting to see which algorithms perform well in a smoothed setting without lookahead.

\paragraph{Using Predictions} More research is needed to find online algorithms that use predictions robustly and consistently. Here, robustness refers to a bounded competitive ratio when predictions are adversarial, and consistency refers to an improved competitive ratio when predictions are accurate \cite{Li2021}. Note that clearly model predictive control-style algorithms can perform arbitrarily poorly when predictions are adversarial.