% !TeX root = ../main.tex
% Add the above to each chapter to make compiling the PDF easier in some editors.

\chapter{Conclusion}\label{chapter:conclusion}

This work, surveyed numerous offline and online algorithms for fractional and integral smoothed convex optimization in single and high dimensions. We evaluated the performance of online algorithms for the application of dynamically right-sizing data centers using traces from various different sources. We found that the discussed online algorithms perform nearly optimally with respect to the observed normalized cost when compared to the dynamic offline optimum. Moreover, we observed that the normalized cost achieved by the online algorithms is robust to changes in the data center model. In our analysis, changes in the data center model only impacted the possible cost reduction with respect to the static offline optimum. We also observed that the difference between fractional and integral solutions is negligible in the application of right-sizing data centers. Further, we discussed how algorithms can use predictions of future loads.

Finally, we made our implementation of the discussed algorithms available. This implementation can be used easily to test the performance in other applications and to compare the performance between algorithms. Our data center model balances energy cost and revenue loss. It can easily be tuned using past data to be safe (i.e., provide enough computing resources for incoming loads), fulfill service level agreements, and provide significant cost savings.

\section{Future Work}

\paragraph{Other Applications} Our implementation separates the model layer from the problem and algorithm layer. Therefore, it is very straightforward to test the empirical performance of the discussed online algorithms in other application areas. In \cref{section:introduction:related_work}, we have given an overview of some promising applications.

\paragraph{Algorithms for Convex Body Chasing} We have mentioned in \autoref{chapter:introduction} that smoothed convex optimization and convex body chasing are equivalent. Therefore, an interesting research project would be to extend our library of implemented algorithms by the known algorithms for convex body chasing to compare their empirical performance. In particular, the $\mathcal{O}(d)$-competitive algorithm obtained by \citeauthor*{Argue2019}~\cite{Argue2019} is highly relevant for the application of right-sizing data centers as it does not impose a restriction on cost functions beyond their convexity.

\paragraph{Dynamic Bounds and Dimensions} In practice, the number of available servers (and even server types) in a data center is likely to change over time. An unexplored area of research is how the discussed approaches for online algorithms can be extended to a setting where the bounds on the decision space $\mathcal{X}$ and the dimension of $\mathcal{X}$ are allowed to change over time. \citeauthor*{Albers2021_2}~\cite{Albers2021_2} discuss how their offline algorithm solving the multi-dimensional integral case can be extended to a setting with time-dependent bounds.

\paragraph{Optimal Valley Filling} In \cref{section:case_studies:method:alternatives}, we have discussed the advantages and disadvantages of valley filling, i.e., scheduling low-priority tasks during periods of low loads to reduce the peak-to-mean ratio. Therefore, an interesting problem is finding optimal server configurations and job scheduling times such that the operating costs and switching costs of the data center are balanced with the revenue loss of delaying specific jobs. This problem extends smoothed convex optimization in the data center setting by allowing for incoming loads to be postponed to a later time slot.

\paragraph{Optimal Assignments of Jobs to Servers} Smoothed convex optimization determines the optimal assignment of jobs to a collection of servers of the same type. In \cref{section:application:dispatching}, we have seen that an optimal dispatching rule is to distribute jobs across all servers evenly. However, this approach does not have to be optimal in practice as job arrival times may vary and jobs are discrete, i.e., it may not be possible to distribute them evenly. Therefore, an interesting problem is determining an optimal scheduling of jobs that accounts for the approximations of smoothed convex optimization.

\paragraph{Long-Running Jobs} An vital research problem is to extend the data center model we discussed in \cref{chapter:application} to support jobs with a longer runtime than the length of a time slot. In our case studies from \cref{chapter:case_studies}, we have seen that for a time slot length on the order of tens of minutes to an hour, a significant fraction of jobs (in practical scenarios more than 50\%) fulfill this criterion. This can be achieved by ``memorizing'' which jobs have not been completed yet and ensuring that enough servers of each type are active in the following time slot so that jobs do not have to be rerouted to a server of a different type.

\paragraph{Gang Scheduling Requirement} A similar problem is to extend the model to allow for mutual job requirements. For example, requirements ensuring jobs are processed simultaneously or on servers of the same type.
